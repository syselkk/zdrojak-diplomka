\probbatch{1}
\probno{1}
\probname{tlustý papír}
\proborigin{Karel žere papír.}
\probpoints{2}
\probauthors{kolar}
\probsolauthors{polednikova}
\probsolvers{140}
\probavg{1,71}
\probtask{%
Odhadněte tloušťku papíru~A4, pokud znáte jeho plošné rozměry, gramáž
a~hustotu (jak obecně, tak číselně). Potřebné údaje si vyhledejte (či
správně odhadněte) pro běžný kancelářský papír.}
\probsolution{%
Abychom spočítali tloušťku papíru, budeme se na něj muset podívat na jako opravdu tenký
kvádr. Tloušťku označíme~$c$ a~ostatní dva rozměry, tedy délku a~šířku, $a$~a~$b$.
Jde o~kvádr, takže umíme jednoduše vyjádřit jeho objem
\eq{
	V= abc\,.
}
Objem samotný neznáme, ale zadání nám napovídá, že můžeme použít ještě gramáž,
tedy plošnou hustotu (v~jednotkách~$\jd{g.m^{-2}}$) a~hustotu. Pro objem platí
\eq{
	V= \frac{m}{\rho}\,.
}
a~pro plošnou hustotu platí
\eq{
	\sigma = \frac{m}{S} = \frac{m}{ab}\,.
}

Vztahy dáme do rovnosti a~upravíme do finálního obecného vztahu pro~$c$
\eq{
	\sigma = \frac{m}{ab} = \frac{V\rho}{ab} = \frac{abc\rho}{ab} \ztoho c =
	\frac{\sigma}{\rho} \,.
}

S~číselnými hodnotami dopadneme následovně: gramáž může být různá, kancelářský
papír má často~$"80 g.m^{-2}"$. Hustotu běžného kancelářského papíru zjistíme třeba
na internetu\footnote{\url{http://wiki.answers.com/Q/What_is_the_density_of_paper}}, zde
použitá hodnota je~$"0.86 g.cm^{-3}"$. Rozměry papíru nakonec ani nebudeme
potřebovat. Po dosazení nám výsledná tloušťka vyjde~$"1e-4 m"$.}