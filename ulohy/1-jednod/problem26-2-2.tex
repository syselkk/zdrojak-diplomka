\probbatch{2}
\probno{2}
\probname{hollow Earth}
\proborigin{Karel neustále vymýšlí nerealistické myšlenkové experimenty}
\probpoints{2}
\probauthors{kolar}
\probsolauthors{barta}
\probsolvers{73}
\probavg{1,81}
\probtask{%
Kdyby se všechna hmota Země vzala a~přemodelovala se na kulovou
slupku o~tloušťce~$d = "1 km"$ (se stejnou hustotou), jaký by tato nová
\uv{Země} měla vnější poloměr?
Jaké by bylo gravitační zrychlení na jejím vnějším povrchu?}
\probsolution{%
Uvažujme poloměr Země~$R\_Z="6\,378 km"$. Objem koule s~tímto poloměrem je
\eq{
	V\_Z=\frac{4}{3}\pi R\_Z^3\,.
}
Objem kulové slupky o~vnějším poloměru~$R\_S$ a~tloušťce~$d$ lze přesně vyjádřit
jako
\eq{
	V\_S=\frac{4}{3}\pi\left[R\_S^3-(R\_S-d)^3\right]
	=\frac{4}{3}\pi\(3R\_S^2d-3R\_Sd^2+d^3\)\,.
}
Pro~$d\ll R\_S$ lze zanedbat členy~$d$ v~druhé a~vyšší mocnině. Požadujeme, aby se
objem slupky a~objem Země rovnaly. Dostáváme tedy rovnost
\eq[m]{
	R\_Z^3&\approx3R\_S^2d\,,\\
	R\_S&\approx\sqrt{\frac{R\_Z^3}{3d}}\doteq"2.9e5 km"\,.
}
Jelikož Země bude stále sféricky symetrická, bude tvořit sféricky symetrické,
tedy centrální, silové pole. V~takovémto poli je zrychlení nepřímo úměrné druhé
mocnině vzdálenosti od středu symetrie, tedy od středu koule.\footnote{A to díky
tomu, že tok gravitačního pole libovolnou uzavřenou plochou splňuje podmínku, které
říkáme \textit{Gaussova věta}.} Ta je v~našem
případě~$R\_S$. Pokud si uvědomíme, že zrychlení na povrchu Země je nyní
\eq{
	g=G\frac{M\_Z}{R\_Z^2}\,,
}
můžeme pomocí tohoto vyjádřit i~nové gravitační zrychlení
\eq[m]{
	g'&=G\frac{M\_Z}{R\_S^2}\,,\\
	g'&=g\frac{R\_Z^2}{R\_S^2}\approx 3G\frac{d}{R\_Z} \doteq"4.6e-3 m.s^{-2}"\,.
}
Zrychlení na povrchu takovéto Země by tedy bylo asi~$2\,000$krát~slabší než na
naší Zemi.}%
