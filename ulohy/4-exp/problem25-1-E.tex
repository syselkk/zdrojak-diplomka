\probbatch{1}
\probno{7}
\probname{brumlovo tajemství}
\proborigin{Karel chtěl, aby medvídci trpěli.}
\probpoints{8}
\probsolauthors{kalasova,pikalek}
\probsolvers{50}
\probavg{4,50}
\probtask{%
Změřte co nejvíc (alespoň 3) fyzikálních vlastností a~charakteristik
želatinových medvídků. Zkoumejte i~rozdíly mezi jednotlivými barvami medvídků
v~pytlíku. Měřit můžete například teplotu tání, Youngův modul pružnosti, mez
pevnosti, savost (změna objemu či hmotnosti medvídka po namočení po nějakou
dobu), hustotu, vodivost, index lomu, rozpustnost (ve vodě, lihu), změnu některé
z~předcházejících vlastností při změně teploty či cokoliv jiného vás napadne.
}
\probsolution{%
Po bližším prozkoumání trhu jsme zjistili, že existují minimálně dva druhy, 
které se na první pohled v~některých vlastnostech zásadně liší. Koupili jsme 
si tedy medvídky Jojo a~Haribo a~podrobili je zkoumání. 
 
\subsubsection{Změna objemu}
 
O~medvídcích je známo, že když se dají do vody, zvětší svůj objem. Medvídka 
jsme považovali přibližně za kvádr a~měřili jeho tři rozměry, se zanedbáním
např. hodně vyčnívajících oušek. Zkoušeli jsme také medvídky ořezat na kvádr, 
ale ke zlepšení přesnosti to moc nepomohlo. Z~těchto údajů jsme vypočítali objem 
medvídka~–~sice přibližný, ale pro podstatu pokusu~–~ukázání zvětšení v~různých 
kapalinách~–~dostatečný. Medvídky jsme dali do vody z~kohoutku, do vody v~ledničce,
do oslazené a~osolené vody a~do nasyceného solného roztoku a~změřili je 
po~$13$ a~$24$~hodinách. Výsledky pro všechny tekutiny jsou v~tabulce~\ref{R25S1UEtbear};
medvídci naložení v~nasyceném solném roztoku měli po $13$~hodinách $"82 \%"$~(Haribo)
a~$"71 \%"$ (Jojo) původní velikosti, po $24$~hodinách $"77 \%"$~(Haribo)
a~$"47 \%"$~(Jojo), tedy zmenšili se.
 
\begin{table}[h]
\setlength{\tabcolsep}{4pt}
\ifyearbook
\footnotesize
\else
\small
\fi
\centering
\begin{tabular}{lld{0}d{0}d{0}d{0}d{0}d{0}d{0}d{0}d{0}d{0}d{0}}
\toprule
  & \multirow{2}{*}{Prostředí} & \multicolumn{1}{c}{$"0 hod"$} & \multicolumn{5}{c}{$"13 hod"$} & \multicolumn{5}{c}{$"24 hod"$} \\
\cmidrule(lr){3-3}
\cmidrule(lr){4-8}
\cmidrule(lr){9-13}
  &  & \multicolumn{1}{c}{\popi{V}{mm^3}} & \multicolumn{1}{c}{\popi{a}{mm}} &%
       \multicolumn{1}{c}{\popi{b}{mm}} & \multicolumn{1}{c}{\popi{c}{mm}} & %
       \multicolumn{1}{c}{\popi{V}{mm^3}} & \multicolumn{1}{c}{\%} &%
       \multicolumn{1}{c}{\popi{a}{mm}} &%
       \multicolumn{1}{c}{\popi{b}{mm}} & \multicolumn{1}{c}{\popi{c}{mm}} & %
       \multicolumn{1}{c}{\popi{V}{mm^3}} & \multicolumn{1}{c}{\%}\\
\midrule
\multirow{5}{*}{\begin{sideways}Jojo\end{sideways}} & normální & 2761 & 34 & 24 & 15 & 12240 & 443 & 38 & 21 & 18 & 14364 & 520 \\
 & slaná & 2761 & 32 & 20 & 13 & 8320 & 301 & 35 & 20 & 14 & 9800 & 355 \\
 & sladká & 2761 & 35 & 24 & 15 & 12600 & 456 & 39 & 25 & 15 & 14625 & 530 \\
  & studená & 2761 & 34 & 20 & 14 & 9520 & 345 & 36 & 23 & 13 & 10764 & 390 \\
  & nasycený $\odot$ & 2761 & 23 & 13 & 7 & 1950 & 71 & 21 & 11 & 6 & 1283 & 47\\
\midrule
\multirow{5}{*}{\begin{sideways}Haribo\end{sideways}} & normální & 2384 & 34 & 20 & 18 & 11603 & 487 & 36 & 22 & 18 & 14256 & 598 \\
 & slaná & 2384 & 28 & 17 & 14 & 6664 & 280 & 30 & 18 & 15 & 8100 & 340 \\
 & sladká & 2384 & 29 & 17 & 17 & 8381 & 352 & 32 & 19 & 18 & 10944 & 459 \\
 & studená & 2384 & 30 & 17 & 16 & 8160 & 342 & 31 & 19 & 16 & 9424 & 395 \\
 & nasycený $\odot$ & 2384 & 18 & 9 & 9 & 1425 & 82 & 18 & 9 & 9 & 1331 & 77\\
\bottomrule
\end{tabular}
\caption{Naměřené hodnoty (sloupec~\% udává poměr nového a~původního objemu)}
\label{R25S1UEtbear}
\end{table}
 
Z~našeho pokusu vyplynulo několik věcí~–~medvídek se v~ledničce moc nezvětší 
a~zůstane i~docela tvrdý. Zjistili jsme taky zásadní rozdíly ve zvětšování 
v~různých kapalinách. Želatinový medvídek i~slaná voda jsou stejnorodé směsi
(roztoky) \uv{něčeho}~--~rozpuštěné látky a~vody~--~rozpouštědla. Výroba medvídků 
začíná právě směsí želatiny a~vody (a~zbytku). Želatina je tvořená řetězovými 
molekulami, které se vzájemně proplétají, a~jak směs chladne a~voda se dostává 
ven, tvrdne a~vznikne medvídek. Haribo zjevně obsahují méně vody než Jojo~–~jsou
tužší a~hůř se deformují. Slaný roztok, na~rozdíl od želatinového, obsahuje 
mnohem méně pevné látky, sůl navíc netvoří žádné propletené řetízky (to je taky 
částečně důvod, proč slaná voda zůstává tekutina a~želatinový roztok tuhne). 
 
\ifyearbook%
\plotfig{problem1-7-data-yb.tex}{Rozpouštění medvídků}{R25S1UEbear}\else%
\plotfig{problem1-7-data.tex}{Rozpouštění medvídků}{R25S1UEbear}%
\fi


Když se dají dva roztoky k~sobě, voda bude mít tendenci se přesouvat z~hustšího 
do řidšího, tento jev se nazývá \emph{osmóza}. Vodu pohání osmotický tlak. Když ponoříte
 medvídka do vody s~málo rozpuštěnými molekulami (např.~do destilované), voda 
se nahrne do medvídka a~zvětší ho. Když dáte medvídka do vody, která obsahuje 
hodně molekul něčeho rozpuštěného (více než medvídek), voda se z~medvídka uvolní. 
Když voda putuje do medvídka, medvídek se zvětšuje; když se z~něj uvolňuje, medvídek 
vypadá stejně. Pokud tedy namícháme hodně slaný roztok, který bude obsahovat víc 
částic než medvídek, medvídek se zmenší kvůli malému osmotickému tlaku.
 
Také jsme zjistili, že medvídci jsou potom hrozně oslizlí a~upadávají jim uši;
ve studené vodě drží trochu víc pohromadě.
 
 
\subsubsection{Index lomu}
 
Změřit index lomu třeba nějaké tekutiny nebo skla není moc složité – vystačíte si 
s~laserem a~použitím Snellova zákona. Bohužel gumoví medvídci mají také tu vlastnost, 
že světlo značně rozptylují, a~to i~tenké kousky. Kvůli značnému rozptylu světla 
se nám nepodařilo index lomu změřit.
 
\subsubsection{Měření měrného elektrického odporu} 
 
Další měření, které jsme prováděli, bylo měření elektrického odporu.
 
Měrná elektrická vodivost je
\eq{
    \varrho = \frac{R S}{l}\,,
}
kde $R$~je odpor medvídka průřezu~$S$ a~tloušťky~$l$, přičemž~$R = {U}/{I} $, kde~$U$~je napětí 
zdroje, ke~kterému je medvěd celou svou plochou připojen, a~$I$~je proud medvědem 
procházející. Tedy
\eq{
    \varrho = \frac{U S}{l I}\,.
}
 

Postup měření byl následující: do elektricky izolovaných čelistí svěráku byly umístěny 
dvě hliníkové desky a~mezi ně byl vložen medvěd tak, aby utahováním svěráku docházelo 
k~jeho deformování mezi rovnoběžnými měděnými destičkami. Medvěd byl takto stlačen na 
tloušťku~$l = "(1.00\pm0.05) mm"$, přičemž se mezi destičkami 
velmi roztáhl do strany. Při této tloušťce již bylo možné zanedbat nerovnost povrchu 
medvěda, jelikož byl z~obou stran celou plochou přilepen k~deskám.

Destičky byly později připojeny ke stabilizovanému zdroji napětí~$U = "12 V"$ a~byl změřen proud~$I$
procházející medvědem. Dále bylo třeba změřit i~plochu~$S$, kterou se medvěd dotýká
 desek. Z~toho důvodu byla po měření proudu jedna z~desek obarvena barvou a~zdeformování
 medvěda na požadovanou tloušťku opakováno. Dále byl medvěd od desky opět odlepen a~na 
desce byl jasně viditelný jeho původní obtisk, jehož obsah byl poté změřen pomocí spočítání 
čtverečků milimetrového papíru. Z~několika měření byla určena průměrná hodnota této plochy
 na~$\overline{S} = "2.6e-3 m^2"$. Stejně tak hodnota měřeného proudu
 se u~jednotlivých medvídků příliš nelišila, průměrně byla asi~$\overline{I} = "6.2e-5 A"$.
Z~těchto naměřených hodnot byl poté podle výše uvedeného vztahu určen měrný elektrický odpor
na $"5.0e5 \ohm.m"$.
 
\subsubsection{Pevnost} 

Medvídek je (alespoň se ze začátku snaží být) pevná látka. Jednou z~možných charakteristik 
pevných látek je mez pevnosti, která vyjadřuje odolnost látky vůči vnějším silám. 
Je to nejvyšší hodnota normálového napětí~$\sigma\_n$, kdy látka ještě drží pohromadě, 
není porušená nebo přetržená. Normálové napětí lze určit jako podíl deformující síly~$F$ 
a~kolmého průřezu~$S$ medvídka na začátku:
\eq{
    \sigma\_n = \frac{F}{S}.
}
 
Při měření jsme medvídkovi změřili obvod a~z~něj vypočítali plochu; upevnili jsme ho 
do stojanu a~zavěšováním závaží zjišťovali, při jak velké síle dojde k~přetržení. 
Bohužel medvídek je z~příliš kluzkého a~špatně upevnitelného materiálu, které nedovoluje
 určit hledané konstanty příliš přesně~–~výsledkem našeho snažení je tedy řádový odhad. 
Mez pevnosti pro medvídka Haribo je řádově $"250 kPa"$ a~$"65 kPa"$ pro Jojo. 
Jojo medvídci mají mez pevnosti menší, to jde poznat i~bez měření~–~Jojo medvídek jde 
roztrhnout lépe.

\ifyearbook
\else
\subsubsection{Komentář k~řešení}

Téma experimentální úlohy bylo zřejmě dost zajímavé, protože vašich řešení došlo nemálo. Měřilo 
se hodně věcí. Vedle \uv{profláklých} vlastností jako hmotnost, objem, hustota, odpor, vodivost,
změna objemu, modul pružnosti, rozpustnost, mez pevnosti, savost, index lomu a~teplota tání,
se měřily i~exotické vlastnosti jako faktor smykového tření, zmáčknutelnost, moment 
setrvačnosti, chuť nebo rozložení barev v~pytlíku. Zvlášť nás potěšilo originální řešení Tomáše 
Axmana, který měřil absorpci světla pomocí slunečního panelu. Je trochu škoda, že málokdo 
se zamyslel nad tím, proč mu výsledky vychází tak, jak vychází. Medvídci tedy opravdu
trpěli – byli natahováni, mačkáni, řezáni na filety, namáčeni do vody, různě sytých slaných roztoků,
Alpy, octa, lihu, Sava nebo Coca-Coly i~strkáni do mikrovlnky. Medvídci taky rádi chytají 
plíseň a~rozpouští se rychleji, než se provede měření. A~někdo snědl medvídky dřív, než stihl
měření dokončit. Takže medvídci i~chutnali. 
\fi
}
