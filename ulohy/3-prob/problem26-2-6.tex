\probbatch{2}
\probno{6}
\probname{gravitace si žádá větší slovo}
\proborigin{Karel zase v~zajetí astrofyziky.}
\probpoints{5}
\probauthors{kolar}
\probsolauthors{kocak}
\probsolvers{54}
\probavg{1,91}
\probtask{%
Co kdyby se \uv{přes noc} změnila hodnota gravitační konstanty na
dvojnásobek a~přitom by zůstaly zachovány ostatní fyzikální
konstanty na původních hodnotách? A~co kdyby se zvětšila stokrát?
Rozepište se o~různých aspektech~--~zejména o~životě na Zemi
a~drahách vesmírných objektů.}
\probsolution{%
Základnom riešenia bolo uvedomiť si, kde všade (v~ktorých javoch) sa objavuje
gravitačná konštanta~$G$, či už priamo alebo nepriamo. Napríklad v~gravitačnom
zákone sa vyskytuje priamo, kde
\eq{ |\vect F_G| = G \frac{M_1 M_2}{r_{21}^2} }
a~v~rovniciach pre šikmý vrh je nepriamo zahrnutá v~gravitačnom zrýchlení~$g$, kde
\eq[m]{
	x &= x_0 + v_{0,x} t \,,\\
	y &= y_0 + v_{0,y} t - \frac{1}{2} g t^2 \,,
}
kde
\eq{
	g = \frac{G M\_{Zem}}{R\_{Zem}^2}\,.
}

Skôr než začneme rozoberať konkrétne prípady treba povedať ešte dve veci. Javy,
ktoré sme tu uviedli, zďaleka nebudú všetky. Pôjde o~najvýznamnejšie, ktoré nás
napadli. Druhá vec, v~pátraní súboru %formulácia
základných konštánt, ktoré priamo určujú konštanty vo všetkých zákonoch známej
fyziky, sa zúžil počet na zopár konštánt (medzi nimi je napríklad rýchlosť svetla,
Planckova konštanta i~gravitačná konštanta), medzi ktorými sa zatiaľ nepodarila nájsť
previazanosť, čo však nevylučuje, žeby sa časom mohla nájsť. Budeme predpokladať,
že sú nezávislé.

Uvažujme zmenu gravitačnej konštanty~$k$-násobkom
\eq{
	G' = kG \,,
}
kde $G$~je pôvodná gravitačná konštanta, $G'$~je nová gravitačná konštanta a~$k$~je
bezrozmerné číslo.

Prvá zjavná vec, ktorá zo zmenou~$G$ prichádza, je zmena gravitácie a~už spomínaný
šikmý vrh na povrchu Zeme. Zo vzťahu pre gravitačné zrýchlenie dostaneme,
že sa~$k$-násobne zväčší
\eq{
	g'=\frac{G' M\_{Zem}}{R\_{Zem}^2} = \frac{kGM\_{Zem}}{R\_{Zem}^2}=kg\,.
}
Predstavme si malý kanón, ktorý strieľa gule priamo nad seba (vojensky neužitočný
kanón). Deň pred zmenou letela guľa do výšky~$h$ a~celý pád jej trval čas~$t$. Keď
riešime tento jednoduchý problém, tak dostaneme v~závislosti od počiatočných
podmienok vzťahy
\eq{
	h = \frac{v_0^2}{2 g}\,, \qquad t = \frac{2 v_0}{g} \,.
}
Na druhý deň nastala zmena konštanty. Síce kanón dodal guli rovnakú kinetickú
energiu (a~tým pádom i~hybnosť a~rýchlosť), ale namerali sme výšku~$h'$ a~čas~$t'$
\eq{
	h' = \frac{v_0^2}{2 g'} = \frac{h}{k}\,, \quad
	t' = \frac{2 v_0}{g'} = \frac{t}{k} \,.
}
Pri šikmom vrhu kanónom je maximálny dostrel dosiahnutý pod uhlom~$"45\dg"$
a~dostrel~$d$~je
\eq{
	d = \frac{v_0^2}{g} \,.
}
Asi už nikoho neprekvapí, že po zmene konštanty bude nový dostrel~$d'$ $k$-násobne kratší
\eq{
	d' = \frac{v_0^2}{g'} = \frac{d}{k} \,.
}
Tak vidíme, že pri hodoch sa $k$-násobne skrátia časy hodov, maximálne výšky
i~dostrely pod konštantným uhlom.

Zábavnejšie to však je v~prípade vesmírnych obežníc, a~to či už sa týka nášho
Mesiaca alebo planét Slnečnej sústavy. Vo všeobecnosti podľa 1. Keplerovho zákona
sa objekty v~radiálnom gravitačnom poli pohybujú po kužeľosečkách. Pohyb po
kružniciach je iba jeden špeciálny prípad rýchlosti a~vzdialenosti od Slnka a~prakticky
nedosiahnuteľný, keďže zo všetkých možných rýchlosti tomu zodpovedá práve jediná
hodnota rýchlosti. Pohyby planét sú síce približne kružnicové, ale fakticky ide o~elipsy
s~malou výstrednosťou/excentricitou (sploštenosťou dráhy). Pre jednoduchosť však
môžeme predpokladať pred zmenou gravitačnej konštanty pohyb planét po
kružniciach. Pred zmenu je potom vzťah medzi rýchlosťou
planéty~$v_1$~a~vzdialenosťou od Slnka~$r_1$ (z~rovnosti gravitačnej a~dostredivej
sily)
\eq{
	v_1^2 = \frac{G M\_S}{r_1} \,.
}
Po zmene gravitačnej konštanty majú všetky planéty pôvodné rýchlosti (to znamená
rovnaká veľkosť i~smer = kolmé na spojnicu so Slnkom) a~v~novom poli sa budú
pohybovať všeobecne po kužeľosečkách. Ak sa gravitačná konštanta zväčší, začne na
ne pôsobiť väčšia dostredivá sila, ako je potrebná na udržanie na kruhovej dráhe,
a~preto sa budú pohybovať po elipsách. Jeden vrchol (afélium) bude v~mieste, kde sa
nachádzali, keď nastala zmena konštanty (lebo rýchlosť je kolmá na spojnicu so
Slnkom iba vo vrcholoch elipsy a~od tohto bodu sa planéty pohybujú bližšie k~Slnku).
Je jasné, že potom druhý vrchol (perihélium) je najbližšia vzdialenosť, na ktorú sa
dostali k~Slnku. Zo zákona zachovania momentu hybnosti a~energie vieme túto
vzdialenosť vypočítať
\eq[m]{
	m v_1 r_1 &= m v_k r_k\,,\\
	\frac{m v_1^2}{2} - \frac{G' M\_S m}{r_1} &= \frac{m v_k^2}{2} - \frac{G' M\_S
	m}{r_k} \, .
}
Po dosadení dostaneme takúto kvadratickú rovnicu
\eq{
	\frac{r_k^2}{r_1^2} (2 k-1) + \frac{r_k}{r_1} (-2k) + 1 = 0 \,.
}
Okrem afélia dostávame aj druhé riešenie
\eq{
	r_k = \frac{r_1}{2 k-1}\,.
}
Teraz na základe tohto výsledku môžeme povedať tieto skutočnosti. Ak by klesla
gravitačná konštanta na viac ako polovicu ($0 < k< 0,5$), tak všetky planéty budú
mať dostatočnú rýchlosť na odlet od Slnka. V~tabuľke~\ref{R26S2U6_tab1} je vidno pre
rôzne~$k$ rôzne najmenšie vzdialenosti planét od Slnka.

% todo: v ročence je moc široká, zatím zkrácena
\begin{table}[h!]
	\centering
	\begin{tabular}{r*{8}{r@{,}l}}
		\toprule
		 \multicolumn{1}{c}{planéta} & \multicolumn{2}{c}{Merkúr} &
		 \multicolumn{2}{c}{Venuša} & \multicolumn{2}{c}{Zem} &
		 \multicolumn{2}{c}{Mars} & \multicolumn{2}{c}{Jupiter} &
		 \multicolumn{2}{c}{Saturn} & \multicolumn{2}{c}{Urán} \\
		 %& \multicolumn{2}{c}{Neptún}
		\\
		\midrule
		$r_1 \, [\rm{AU}] $ & 0&39 & 0&72 & 1&0 & 1&52 & 5&20 & 9&54 & 19&18 \\
		% & 30&06 \\
		$r_2 \, [\rm{AU}] $ & 0&13 & 0&24 & 0&33 & 0&51 & 1&73 & 3&18 & 6&39 \\
		% & 10&02 \\
		$r_{100} \, [\rm{AU}] $ & 0&002 & 0&004 & 0&005 & 0&008 & 0&026 & 0&048 &
		0&096 \\ % & 0&151 \\
		\bottomrule
	\end{tabular}
	\caption{Najmenšie vzdialenosti planét od Slnka pre rôzne~$k$.} % prosím opravte to, neumím moc slovensky, Pikoš
	\label{R26S2U6_tab1}	
\end{table}

Najskôr si budú v~dráhe prekážať susedné planéty. Pri zvyšovaní už pri~$k = "1.19"$
nastáva prekryv možných oblastí stretnutia medzi Zemou a~Venušou. Pre~$k=2$ sa
jedine neprekrývajú dráhy Jupitera a~Marsu. O~zábavu sa postará pásmo planétok,
ktoré je pekne rozložené medzi Marsom a~Jupiterom a~ktoré bude mať perihélium
približne~$"0.6 AU"$. To znamená, že by sme sa mohli pripraviť na deštrukčnú
vesmírnu prestrelku. Pri eliptických dráhach sa pohybujú planéty v~podstatne väčšom
rozsahu vzdialeností od Slnka, čím sa podstatne zvýši vplyv vzájomnej gravitačnej
interakcie planét. Takže by sme mohli byť skôr, či neskôr svedkom zrážky planét alebo
vyhodenia planéty zo Slnečnej sústavy niektorou z~väčších planét. Tak či onak by to
boli pre Zem časy nepekné (pekelné alebo mrazivé). Pri pôvodnej gravitačnej konštante
je polomer Slnka je~$"0.004\,6 AU"$. Zvýšením gravitačnej konštanty sa polomer Slnka
zmenší, ale stále bude mať Slnko so svojou pre Zem nebezpečnou atmosférou rozmer rádovo
tisíciny astronomickej jednotky. Pre~$k=100$ už je jasné, že
planéty Merkúr, Venuša a~Zem budú míňať Slnko v~tesnej blízkosti alebo narazia na jeho
povrch. V~perihéliu by sa Zem usmažila pri teplote cca~$"3\,700 \C"$ (už by
sme sa nemohli sťažovať na slabé leto), deň by trval 18 hodín a~noc 6 hodín. Pri takej
teplote by bola Zem úplne roztopená (až na diamanty a~grafit, ktoré by čoskoro
zhoreli vo vzduchu) a~bola by to lietajúca kvapka magmy (vhodnejší výraz kvapa
magmy). Kolízia s~inými planétami by bola otázka času.

Ďalším javom, ktorý by gravitačná konštanta skomplikovala život na zemi sú kapilárne
javy. Výška~$h$, do ktorej vystúpi kvapalina v~kapiláre, je
\eq{
	h = \frac{2 \sigma \cos{\alpha}}{r \rho g}\,,
}
kde $\sigma$~je povrchové napätie, $\alpha$~je styčný uhol, $r$~je polomer kapiláry,
$\rho$~je hustota kvapaliny a~$g$~je gravitačné zrýchlenie. Čiastočne funguje transport
vody v~pôde, a~potom i~v~úzkych cievnych zväzkoch, na základe kapilarity, čím sa
zabezpečuje transport látok. Zmenou gravitačnej konštanty sa zníži výška vzlínania na
\eq{
	h' = \frac{2 \sigma \cos{\alpha}}{r \rho g'} = \frac{h}{k}\,.
}
Tým sa značne skomplikuje transport látok najmä vysokým rastlinám, stromom.

Keďže aspektov, kde sa to odrazí, je skutočne veľa, uvedieme iba zopár príkladov bez
podrobnejšej analýzy.
\begin{compactitem}
\item Zmenou gravitačnej konštanty bude vzduch priťahovaný silnejšie, čím sa zvýši
hustota a~tlak vzduchu pri povrchu Zeme.
\item Aby mohla byť voda v~potrubí vytlačená do vyšších poschodí, potrebujeme na
to podľa Bernoulliho rovnice tlak. Zmenou gravitačnej konštanty potrebujeme dodať
vode väčšiu potenciálnu energiu, teda budeme potrebovať väčší tlak, ktorý by
vykonával prácu.
\item Stavby sú síce navrhnuté tak, aby vydržali viac ako maximálnu záťaž (takže
$k=2$~by asi prežili), ale pri určitej hranici sa prekročí medza pevnosti materiálu
a~stavby sa zrútia (budovy, mosty, \dots).
\end{compactitem}

\ifyearbook \else
Úloha bola hodnotená podľa hĺbky analýzy a~počtu javov, kde sa zmena odrazí.\fi}
