\probbatch{5}
\probno{6}
\probname{světelný meč}
\proborigin{Organizátoři se inspirovali vlastní legendou.}
\probpoints{5}
\probsolauthors{kolar}
\probsolvers{21}
\probavg{2,81}
\probtask{%
Navrhněte konstrukci světelného meče, aby byl sestrojitelný za současného poznání 
vědy a~techniky a~přitom vypadal i~fungoval podobně jako ten autentický ze Star~Wars.}
\probsolution{%
\subsubsection{Úvod ke světelným mečům}
Uvědomíme si nejdříve, co světelný meč je. Ti, co znají Star~Wars, to jistě 
dobře ví, a~ti, kteří nikdy tuto ságu nesledovali, mohou snadno najít
informace například na internetu.%
\footnote{\url{http://cs.wikipedia.org/wiki/Světelný_meč}}
\footnote{\url{http://en.wikipedia.org/wiki/Lightsaber}}

Základem světelného meče je cca~$"30 cm"$ dlouhý jílec, který má obvykle kovový vzhled, 
následovaný v~zapnutém stavu čepelí, která je cca~$"1.3 m"$ dlouhá a~je 
tvořena jakousi \uv{zářivou energií}~–~v~podstatě světlem.
\uv{Energie} může mít různou barvu, nejčastější
bývá modrá, zelená a~červená. Objevily se i~vzácnější světelné meče žluté,
fialové a~dokonce i~černé%
\footnote{\url{http://www.youtube.com/watch?v=tw-rYjYnAzE&feature=related}}
 (zrovna svítící černé barvy se současnou fyzikou asi 
nedosáhneme).
V~rámci všech filmů a~příběhů se vyskytly i~exotické meče, které měly například
dvě spojené čepele, ale to byly de facto dva spojené meče. Zde udané délky jsou
jenom typické střední délky, protože např. v~III.~epizodě je vidět, že 
Yoda používá kratší meč než Darth Sidious. Pro nás je ale hlavně důležitý fakt, 
že je délka meče konečná. Ve vypnutém stavu má pak meč rozměr rovný pouze 
rozměru jílce.

Důležité jsou pak další vlastnosti meče ve filmech. Zejména se s~ním dá bojovat, 
tzn. pokud se dva meče zkříží v~rámci akční scény, tak se o~sebe zastaví. Jinak
by to asi nebyl moc dobrý meč. Zajímavou vlastností je, že světelný meč
dokáže rozřezat téměř všechno, až na nějaké exotické materiály jako phrik,
cortosis a~mandalorianskou ocel. To jsou též sci-fi materiály, takže budeme
chtít, aby náš meč řezal prakticky všechno. Dalším omezením je, že chceme, aby
meč byl jak sečnou, tak i~bodnou zbraní (např. ukázka bodného použití%
\footnote{\url{http://www.youtube.com/watch?v=Ku5zkPdKOBY}}
v~čase 3:10 a~posléze v~čase 4:47 i~sečného).

Ze stejné nahrávky by mohlo být patrné, že použití meče na lidskou tkáň vede
k~automatické kauterizaci (zcelení ran pálením).
Malé rozměry jílce se stávají další komplikací pro uschování dostatečně silného
energetického zdroje. Zdá se také, že místo jedno- či dvoustranné čepele má 
válcově symetrickou čepel. Dokáže odrážet střely z~blasteru. Při souboji
vydává specifický zvuk.

Velice nepříjemný technický oříšek je jeho použitelnost vždy a~všude. Dá se 
použít jak v~jakékoliv atmosféře, tak ve vakuu nebo pod vodou (např. Kit Fisto 
v~čase kolem 1:00 ve~videu%
\footnote{\url{http://www.youtube.com/watch?v=n3wLesNq4LI}}).
Navíc se dá zapnout téměř okamžitě a~není příliš křehký. Sice se dá rozbít, ale
musí se jednat o~opravdu velice agresivní pád nebo o~jeho vyslovené rozříznutí.

Michio Kaku, teoretický fyzik a~fanoušek sci-fi, se pokusil v~rámci seriálu
Sci-fi Science odhalit, jakou má současná věda možnost zkonstruovat světelný meč.
Seriál můžete shlédnout na~YouTube.%
\footnote{\url{http://www.youtube.com/watch?v=xSNubaa7n9o}}
Řešení se dále místy odvolává právě na tento pořad, snaží se ho rozšířit 
a~upozornit na další technické problémy a~možnosti.

\subsubsection{Energetický zdroj}
Dostatečně kompaktní energetický zdroj pro zbraň je jedním z~klíčových problémů 
konstrukce. Na energetickém zdroji asi rovnou skončí naše vize mít opravdu
silný meč, co rozřeže téměř cokoliv. Jsou ale určité cesty, kterými by snad 
mohl být napájen, i~když ještě dnes nejsou úplně ve stavu, kdy by se daly rovnou
použít. Vměstnat výkon nějaké menší elektrárny do jedné ruční zbraně totiž není
nic dvakrát jednoduchého.

Technický výdobytek, který navrhuje Michio Kaku, je baterie z~uhlíkových nanotrubiček. 
Uhlíkové nanotrubičky vedou elektrický proud a~mohly by se tak použit jako 
desky miniaturních kondenzátorů. Vzhledem k~tomu, že by se takových miniaturních 
destiček o~šířce nanotrubičky vešlo do malých rozměrů velmi vysoké množství,
tak by po nabití takový kondenzátor mohl sloužit jako zdroj energie naší zbraně.
Má to ovšem určité mouchy, o~kterých Kaku raději nemluví. Evidentní je, že by
se muselo podařit vytvořit vždy vrstvu nanotrubiček a~pak mezi ně dát
nějaký co nejlépe izolující materiál. U~takto malých kondenzátorů by se nejspíš 
stal problémem i~tunelový proud mezi sousedními deskami. Pokud by se tyto
problémy podařilo překonat, tak by to byl asi téměř ideální zdroj díky svojí
velké skladnosti a~přenosnosti.

Pokud bychom chtěli mít zdroj energie v~rukojeti, tak nám opravdu nezbývá nic 
než hledat nějaké nanotechnologie. Jaké různé nové druhy baterií se v~dnešní
době vyvíjí, si můžete přečíst např. v~magazínu.%
\footnote{\url{http://www.chip.cz/clanky/trendy/2011/05/vykonne-baterie-zitrka/article_view?b_start:int=0&-C=}}
Všechny mají pro nás ale dost nedostatečnou kapacitu. Proto můžeme uvažovat
o~tom, jak si pomoci jinak. Docela hloupá alternativa by byla nějaká baterie, 
kterou by měl bojovník na sobě, například v~batohu na zádech, a~bylo by ji 
potřeba připojit k~meči před použitím. Hloupá je, protože by to omezovalo
pohyb nositele meče, nevypadalo by to jako ve Hvězdných válkách a~navíc by
se meč nedal házet zapnutý, což občas jeho nositelé používají. Mohli bychom si
pomoci solárními články v~oblecích rytířů, ale to nebude zase příliš velká 
pomoc. Má stejný problém jako uložení baterie a~navíc světelný meč jde použít 
i~ve tmě, kde ho právě někdy používají Jediové místo baterky.

Zajímavou alternativou by bylo využít nápad, který měl již Nicola Tesla. 
Pokud bychom umístili do dostatečné blízkosti boje naši elektrárnu, co by 
vysílala elektřinu, či spíše energii, do okolí ve formě elektromagnetických 
vln, a~zařídili 
bychom to tak, že by ji meč dokázal sbírat v~průběhu boje, tak bychom měli 
vyhráno. Ale zase tím přicházíme o~jakousi efektivnost jeho použití -- musíme
si s~sebou vozit elektrárnu. Problém by mohl nastávat i~u~příjmu energie, 
protože by nesmělo záviset na poloze meče. Musel by přijímat nějaký stabilní
minimální výkon, což nás nabádá k~tomu rozmístit elekráren více s~různě 
polarizovanými vysílači. A~pokud bychom potřebovali energie opravdu hodně,
pak můžeme narazit na problém, jak neugrilovat našeho bojovníka jenom
samotným elektromagnetickým zářením.

Dále už předpokládejme, že jsme energetický problém vyřešili, i~když to tak 
zcela není.

\subsubsection{Laserový meč}
Asi každého fanouška napadne, že když se říká světelný meč, tak by měl být ze 
světla, a~tedy nejspíše laseru. Vzhledem k~tomu, že i~autoři tvrdí, že 
uvnitř rukojeti meče se skrývá krystal, který je nejdůležitější součástí meče
a~který dává meči jeho barvu a~další vlastnosti, pak nás to směřuje právě
k~laserům. Má to ovšem hned na první pohled zásadní chybu. Laserový paprsek 
může být sice silnou zbraní a~řezat cokoliv, ale není konečný a~nemůže sám o~sobě 
fungovat jako meč, protože se při souboji meče prostě minou a~nemůže tak sloužit
k~obraně vlastníka. Má ale tu výhodu pro fanoušky, že může mít prakticky 
jakoukoliv barvu (kromě černé a~hnědé), i~když v~bezprašném prostředí či za 
silného denního osvětlení paprsky vlastně vůbec neuvidíme, takže zase nebude 
vypadat tak dobře.

Konečnost meče a~společně s~tím i~jeho možné použití v~boji bychom mohli zařídit
výsuvným zrcátkem, které by bylo upevněno na velice pevné zásuvné tyčce.
Tyčka by byla uprostřed meče a~byla by dokola obklopená svazky laserového záření.
Tyčka by nesměla být prakticky vůbec ohebná, protože jejím prohnutím by se změnila
poloha zrcátka a~to by mohlo odrazit smrtelně nebezpečné záření zpět k~ruce 
držitele meče a~při velkém průhybu by opět šlo záření úplně mimo zrcátko.
Navíc její materiál by musel být laseru-odolný (obdobně jako zmíněné zrcátko). 
Takže rovnou ho budeme muset stavět tak, aby nerozřezal úplně všechno.
S~odrazy by vůbec byl problém. Při rozřezávání by náhodný odraz mohl zranit 
náhodné kolemjdoucí, protivníka i~držitele meče. Odraz laseru zpátky do zdroje
zvyšuje nároky na kontrolu síly laserového paprsku uvnitř meče, protože 
bychom přehnanou produkcí laserového záření, které by se nám vracelo po 
optické cestě zpět, mohli zničit krystaly, ve kterých laserování probíhá.
Umístěním zrcátka na konec meče 
jsme se zbavili možnosti použití meče jako bodné zbraně, pokud bychom ho nějak 
nevylepšili.

\subsubsection{Plazmový meč}
Michio Kaku navrhuje konstrukci plazmového meče, ze kterého by proudilo rozžhavené
plazma. Jeho \uv{ostří} by bylo tvořeno keramickým materiálem, který by vydržel velmi
vysoké teploty. Keramika je na druhou stranu nepraktický materiál, protože je 
křehký.

Plazma by se vytvářelo z~okolního vzduchu, který by byl nasáván do hlavice meče 
a~proudil by skrz rukojeť, ve které by se zahříval, ionizoval a~dál putoval
do \uv{ostří}, ve kterém by bylo velké množství malých otvorů a~s~pomocí
elektromagnetického obvodu, cívky uschované v~keramice, by bylo plazma rovnoměrně
distribuováno do okolí čepele. Získali bychom tak meč, který by byl válcově 
symetrický, mohl by sloužit jak jako sečná, tak i~bodná zbraň, a~docházelo by u~něj 
ke kauterizaci ran. Skladnost meče by se zajistila zásuvným 
mechanismem keramické čepele.

Velká nepraktičnost meče je v~omezeném použití jenom v~obvyklé atmosféře.
Ve vakuu by nešel používat určitě, v~jiných hustotách tekutin by pak minimálně
potřeboval nějak seřídit a~upravit, ale rozhodně by se nedal použít jen tak 
jednoduše.

\subsubsection{Chlazení}
S~problémem vysoké spotřeby energie a~potažmo i~použitím vysokých teplot 
u~plaz\-mového meče nám vznikají velké nároky na chlazení jeho jílce. Ve filmech
můžeme sledovat, jak ho drží postavy rukou, což by bylo neuskutečnitelné 
bez nějakého chlazení. U~plazmového meče probíhá svým způsobem aktivní
chlazení natahováním vzduchu z~okolního prostředí, ale u~této konstrukce si
pak budeme úmyslně produkovat další teplo, což chlazení nepomůže.

Stoupající spotřeba energie nám vadí kvůli odporům součástek, kterými poteče 
elektrický proud. V~extrémním případě by se nám mohlo podařit i~součástky 
vypařit. Odpor alespoň některých součástek by se dal anulovat, pokud by se 
podařilo objevit ultra-vysokoteplotní supravodiče supravodivé za pokojových 
teplot. V~současné době známé látky, tzv. vysokoteplotní supravodiče, mají
potřebné vlastnosti při teplotách kolem kapalnění dusíku. Bylo by samozřejmě
možné mít uvnitř chladící systém, který by chladil obvody na nižší teplotu, ale
tím pádem by bylo chlazení ještě složitější.

Pro chlazení potřebujeme nějak odvádět teplo pryč z~jílce. K~tomu je potřeba
nějaké chladící médium. V~případě přítomnosti okolní atmosféry se dá použít 
okolní vzduch či voda. V~případě souboje ve vzduchoprázdném vesmíru narazíme
na další problém. Nejspíše by ale tak jako tak bylo potřeba vyvinout nějaký 
účinný systém chlazení pomocí cirkulace chladící kapaliny, která by se vypařovala
do okolního prostředí a~měla by vysoké latentní teplo varu a~teplotu varu
o~nějaké rozumně nízké hodnotě. Tím zase narazíme na problém s~doplňováním
kapaliny, kterou nikdy Jediové doplňovat nemuseli.

\subsubsection{Závěr}
Dokonalou kopii světelného meče z~Hvězdných válek nejspíš nikdy nebude možné 
vyrobit kvůli velkému množství požadavků, které musí zároveň splňovat. 
Vyrobit kopii, která alespoň vypadá podobně jako meče ve filmech, se dá relativně
jednoduše a~jsou o~tom desítky internetových stránek. Vyrobit něco, co by se mu
funkčně blížilo, je dosti ošemetná věc a~i~když se Michio Kaku v~seriálu tváří,
že nejsme tak daleko od jeho realizace, tak nám v~cestě stojí ještě spousta 
technických problémů. Pokud by ale někdo hodlal věnovat do vývoje světelného
meče pár miliard dolarů, tak věřím, že za deset, dvacet let by mohl mít 
relativně dobře funkční výrobek.

\ifyearbook
\else
Z~hlediska hodnocení a~bodování úlohy budeme brát jako důležité hlavně výčet
co nejvíce fyzikálních vlastností a~zhodnocení jejich možné (ne)realizace
v~nějakém modelu meče.
\fi

% \subsubsection{Komentář k došlým řešením}

}
