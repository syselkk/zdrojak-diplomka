\probbatch{6}
\probno{5}
\probname{běh na přednášku z~eugeniky}
\proborigin{Karel civěl na internet.}
\probpoints{4}
\probsolauthors{hermann}
\probsolvers{24}
\probavg{2,88}
\probtask{%
Aleš sedí pod kopcem u~stanu a~surfuje na internetu na svém tabletu, když tu si
náhle všimne, kolik je hodin, a~uvědomí si, že vlastně chtěl jít na přednášku.
Už je tak pozdě, že bude muset celou cestu běžet a~nebude moct zastavit, ani aby se
vydýchal. Proto se samozřejmě okamžitě rozběhne svou maximální běžeckou rychlostí~$v$
do kopce, který má rovnoměrné stoupání~$\alpha$. Po chvíli (čas~$T$) si ale
uvědomí, že má v~kapse cihlu a~že tu cihlu chtěl nechat u~stanu. Aleš od sebe
umí cihlu hodit jedině rychlostí~$w$. Pod jakým úhlem má cihlu v~tom okamžiku
vyhodit, aby dopadla na kamaráda, co si právě sedl na jeho místo? Může se stát,
že nedohodí? Aleš je hodně rychlý, a~proto neuvažujte jeho reakční dobu
a~ani dobu, kterou vám zabere řešení úlohy.
}
\probsolution{%
\ifyearbook%
\illtoptrue%
\illfig{problem6-5-nakres.eps}{}{R25S6U5_nakres}{}%
\illtopfalse%
\else\fi
\noindent Na začátku sedí Aleš v~počátku souřadného systému. Cihlu odhazuje 
v~bodě $(x_0,y_0)=vT(-\cos\alpha,\sin\alpha)$. Alešova rychlost je $v(-\cos\alpha,\sin\alpha)$ 
a~v~jeho souřadném systému bude odhazovat cihlu rychlostí $w(\cos\beta,\sin\beta)$. V~nehybném 
souřadném systému bude tedy odhazovat rychlostí
\eq{
    (w'_x,w'_y)=(w\cos\beta-v\cos\alpha,w\sin\beta+v\sin\alpha)\,.
}

Trajektorie cihly bude
\eq{
    (x,y)(t)=\big(x_0+w'_xt,y_0+w'_yt-\frac12gt^2\big)\,.
}
Naším úkolem je vyřešit soustavu rovnic $(x,y)(t;\alpha,\beta,T,v,w)=0$ pro neznámé $t$ a~$\beta$, 
zatímco $\alpha$, $T$, $v$ a~$w$ jsou parametry.

Z~kvadratické rovnice $y(t)=0$ dostaneme
\eq{
    t=\frac{w'_y}g+\sqrt{\( \frac{w'_y} g \)^2+\frac{2y_0}g}\,.
}
Dosazením do $x(t)=0$ a~rozepsáním $w'_x$ a~$w'_y$ dojdeme
k~\eq[m]{
    -vT\cos\alpha+(w&\cos\beta-v\cos\alpha) \left[ \frac{w\sin\beta+v\sin\alpha}g + 
    \vphantom{\sqrt{\(\frac{w\sin\beta+v\sin\alpha}g\)^2}} \right.\\ % kvůli výšce levé závorky
    & \left. + \sqrt{\(\frac{w\sin\beta+v\sin\alpha}g\)^2+\frac{2vT\sin\alpha}g} \right] = 0\,.
		\lbl{R25S6U5_rovnice}
}
Několika algebraickými úpravami a~schováním $v$, $w$, $g$ a~$T$ do bezrozměrných $Q=w/v$ a~$A=gT/v$ 
dostaneme
\eq{
    A(1+\cos2\alpha)+4Q\sin(\alpha+\beta)(\cos\alpha-Q\cos\beta)=0\,.
}
Tato rovnice bohužel vede na kvadratickou rovnici v~$\sin\beta$ (nebo $\cos\beta$), jejíž monstrózní 
řešení nám neřekne zhola nic, a~proto se uchýlíme k~numerickému řešení.

Pro tyto účely je dobré se vrátit k~rovnici~\eqref{R25S6U5_rovnice}, do které jsme ještě nezanesli 
neekvivalentními úpravami nesprávná řešení. Po chvíli hraní s~touto rovnicí v~počítači dojdeme k~závěru, 
že dle očekávání má Aleš buď smůlu a~nedohodí, a~nebo má na výběr ze dvou různých úhlů $\beta$. Taktéž se 
ukazuje, že pro numerické účely bude vhodnější nehledat $\beta$ v~závislosti na $\alpha$ (0 nebo 2~řešení), 
ale raději $\alpha$ v~závislosti na $\beta$ (0 nebo 1~řešení). Výsledné grafy (už zase jako $\beta(\alpha)$) 
jsou na obrázku~\ref{R25S6U5_reseni}.

\fullfig{problem6-5-reseni.eps}{Numerické řešení pro $A="0.2";"1";"5"$ (zleva doprava) 
a~$Q="0.2";"0.7";"1.3";"1.7";"2.5";"10"$ (od černé k~šedé). Na ose~$x$ je úhel~$\alpha$; 
na ose~$y$ úhel~$\beta$.}{R25S6U5_reseni}

Numerické řešení potvrzuje intuici. Pro některé kombinace $A$ a~$Q$ lze dohodit pod kopec až pro 
úhly $\alpha\geq\alpha\_{min}$, přičemž pro $\alpha=\alpha\_{min}$ existuje jeden kýžený úhel 
$\beta=\beta\_{min}$, zatímco pro $\alpha>\alpha\_{min}$ existují dva úhly $\beta_1<\beta\_{min}<\beta_2$. 
Čím větší $Q$ (větší $w$), tím menší $\alpha\_{min}$. Čím větší~$A$ (delší~$T$), tím větší~$\alpha\_{min}$.
}
