\probbatch{1}
\probno{4}
\probname{drrrrr}
\proborigin{Jáchym hodil do stroje kuličku.}
\probpoints{4}
\probsolauthors{pulmann}
\probsolvers{27}
\probavg{2,30}
\probtask{%
Mezi dvěma opačně nabitými deskami se sem a~tam odráží vodivá kulička
zanedbatelných rozměrů. S~jakou frekvencí se pohybuje? Napětí mezi deskami
je~$U$. Při nárazu se kulička nabije na náboj velikosti~$Q$ shodný s~polaritou desky.
Koeficient restituce je~$k$.\\
\emph{Bonus:} Odpovídá výkon na tomto rezistoru energetickým ztrátám při nárazech?\\
\emph{Poznámka:} Koeficient restituce je poměr kinetických energií po nárazu a~před ním.
}
\probsolution{%
Keďže sa guľôčka nabije na náboj rovnakej polarity ako doska, do ktorej narazí,
po každom náraze je naďalej urýchlovaná elektrickým poľom. Takto ale
nezískava energiu neustále, po zopár nárazoch sa jej pohyb stane takmer 
periodický. Guľka totiž pri náraze stráca energiu, a~hodnota tejto energie 
závisí od rýchlosti. Čím ide rýchlejšie, tým má väčšiu kinetickú energiu, 
a~teda aj o~viac energie príde. Takto bude získavať energiu z~potenciálového 
rozdielu a~zrýchlovať, až pokiaľ sa nedostane do ustáleného stavu,
v~ktorom pri zrážke stratí rovnakú energiu, akú získa pri prechode medzi 
doskami. Na rovnakej rýchlosti by sa ustálila aj v~prípade, ak by najprv 
išla prirýchlo.
 
Označíme si rýchlosť tesne pred dopadom~$v$ a~po odraze~$u$. Koeficient 
reštitúcie je definovaný ako
\eq{
    k = \frac{\frc{1}{2} m u^2}{\frc{1}{2} m v^2} =  \frac{u^2}{v^2}\,.
}
 
Teraz vieme vypočítať ustálenú rýchlosť, napríklad tú pred dopadom
 
\eq{
    \frac{1}{2}mv^2 - \frac{1}{2}m u^2 = UQ\,.
}
 
Čo vyjadruje, že guľôčka získa na napätí rovnakú energiu, akú stratí pri 
náraze. Rýchlosť $v$ už len vyjadríme
\eq[m]{
    \frac{1}{2}mv^2 (1-k) &=  UQ\\
    v &= \sqrt{\frac{2UQ}{m(1-k)}}\,.
}
 
Ako vyzerá pohyb medzi doskami? Ak sú tieto dosky dostatočne veľké 
v~porovnaní s~medzerou medzi nimi, tak môžeme elektrické pole považovať 
za homogénne, a~takéto pole pôsobi na guľôčku konštantnou silou. Pohyb je 
teda jednoducho rovnomerne zrýchlený. Ak si označíme vzdialenosť dosiek 
$d$, a~uvedomíme si, že priemerná rýchlosť je aritmetický 
priemer $u$ a~$v$ (je to kvôli konštantnému zrýchleniu, premyslite si!), 
čas prechodu medzi doskami bude \eq{
    t = \frac{d}{(u+v)/2} = \frac{2d}{u+v}\,.
}
 
Frekvencia je obrátená hodnota periódy, a~v~našom prípade perióda zahrňuje pohyb 
tam a~spať, teda je dvojnásobok času $t$.
\eq{
f = \frac{1}{2t} = \frac{u+v}{4d}
}
 
Teraz už len dosadíme za rýchlosť 
\eq{
    f =  \frac{1+\sqrt{k}}{4d}v = \frac{1+\sqrt{k}}{4d}\sqrt{\frac{2UQ}{m(1-k)}} 
    = \sqrt{ \frac{UQ}{8md^2}\left(\frac{1+\sqrt{k}}{1-\sqrt{k}}\right) }\,.
}
 
Tu síce vystupujú parametre, ktoré neboli v~zadaní, no jednoduchou 
úvahou zistíme, že tam skutočne majú byť, a~nedajú sa vyjadriť. 
Obe veličiny, hmotnosť aj vzdialenosť dosiek, vieme meniť nezávisle od 
zvyšných zadaných parametrov, takže vieme vyrobiť dve situácie, 
ktoré sa líšia len napr. hmotnosťou guľôčky, ktorá teda nemôže byť 
kombináciou už zadaných veličín ako napätie, prenášaný náboj či 
koeficient reštitúcie.
 
V~bonuse určite zanedbáme všetky straty energie okrem tých, ktoré 
nastávajú pri zrážke, kvôli koeficientu reštitúcie. Tu potom jasne 
vidíme, že výkon, ktorý dodáva zdroj poskytujúci napätie $U$, sa 
mení len na stratový výkon pri nárazoch. Presvedčiť sa o~tom 
dá aj priamo počítaním týchto dvoch výkonov, všetko potrebné už 
máme vyjadrené. Stačí len náboj prenesený za čas $t$ (takto totiž 
vyzerá definícia prúdu) vynásobiť napätím $U$, a~tento výkon porovnať 
so stratou kinetickej energie pri jednom náraze delenou časom $t$, 
aby sme opäť dostali výkon (čo, ako si môžeme všimnúť, je nakoniec 
len rovnica vyjadrujúca rovnosť získanej a~stratenej energie, z~ktorej 
sme vychádzali na začiatku).

Vidíme teda, že ak si všímame len vonkajší výkon a~prúd, správa sa 
takýto rezistor ako skutočný odpor. Skôr ako však bežíme na patentový 
úrad, mali by sme si so sklamaním všimnúť, že jeho voltampérová 
charakteristika nie je priamka. Prúd $I=Q/t=Qf$ závisí od napätia 
nie lineárne, ako by sa na správny rezistor patrilo, ale komplikovanejšie, 
kvôli čomu klesá odpor s~druhou odmocninou napätia. Analógia so skutočným 
rezistorom je teda možná len čiastočne.
}
