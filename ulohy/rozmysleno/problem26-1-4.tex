\probbatch{1}
\probno{4}
\probname{crash testy}
\proborigin{Analphabeth Petr s~ryšavými vlasy}
\probpoints{4}
\probauthors{rysavy}
\probsolauthors{sykoraj}
\probsolvers{101}
\probavg{3,33}
\probtask{%
Mějme dvě auta o~stejné hmotnosti jedoucí proti sobě rychlostí~$v_0$.
V~jaké vzdálenosti musí začít brzdit, aby nedošlo ke srážce? Uvažujte
dva případy, kdy auta jedou proti sobě na rovině a~kdy auta jedou po silnici
se sklonem~$\alpha$. Víte, že oba řidiči začnou brzdit v~týž okamžik
a~velikost brzdné síly každého auta je~$f \cdot N$, kde $N$~je složka
tíhy automobilu kolmá na silnici.}
\probsolution{%
První případ je pouze speciálním případem případu druhého, a~to pro~$\alpha=0$.
Můžeme se tedy rovnou pustit do obecného řešení. Předpokládejme, že jakmile auto
zastaví, už zůstane stát. Spočítejme, jak daleko od sebe by auta musela začít
brzdit, aby zastavila těsně u~sebe. Hledaná vzdálenost tedy bude součtem
brzdných drah.

Rozkladem tíhové síly o~velikosti~$G=mg$ na složku normálovou k~povrchu a~složku
tečnou k~pohybu zjistíme, že~$N=mg\cos\alpha$. Brzdná síla je
tedy~$fmg\cos\alpha$. Nezapomínejme však, že jede-li auto po nakloněné rovině,
je navíc zrychlováno či zpomalováno tečnou složkou tíhy. Celková síla působící
na vozidlo ve směru pohybu je tedy~$fmg\cos\alpha\pm mg\sin\alpha$. Znaménko
ve výrazu určuje směr sklonu silnice a~právě v~něm se budou lišit zkoumaná dvě
auta. A~zrychlení pak bude~$a=fg\cos\alpha\pm g\sin\alpha$.

Pro rovnoměrně zpomalený pohyb zakončený stáním platí vzorec pro výpočet ujeté
vzdálenosti
\eq{
	s=\frac{1}{2}at^2\,,
}
kde $t$~je čas potřebný k~zastavení. Zároveň~$t=v_0/a$. Můžeme tedy upravit na
\eq{
	s=\frac{v_0^2}{2 a}=\frac{v_0^2}{2 g(f\cos\alpha\pm \sin\alpha)}\,.
}
Po sečtení pro obě auta dostáváme celkovou vzdálenost
\eq{
	S=\frac{v_0^2}{2 g}\(\frac{1}{f\cos\alpha+\sin\alpha}
		+\frac{1}{f\cos\alpha-\sin\alpha}\)\,,
}
což lze upravit na
\eq{
	S=\frac{v_0^2}{g}\frac{f\cos\alpha}{(f^2+1)\cos^2\alpha-1}\,.
}
Dosazením~$\alpha=0$ zjistíme, že na rovině je to
\eq{
	S=\frac{v_0^2}{gf}\,.
}}
