\documentclass[a4paper,11pt,oneside]{article}

\usepackage[IL2]{fontenc}
\usepackage[utf8]{inputenc}
\usepackage[czech]{babel}

\usepackage{fales2}
%\usepackage{titulka_a}

\usepackage{geometry} %rozložení stránky

\usepackage{amsfonts,amsmath,amssymb}

\usepackage{graphicx}
\usepackage{color}
\usepackage{epstopdf}

\usepackage{booktabs} %tabulky
\usepackage{array}
\usepackage{dcolumn} %zarovnání tabulek podle desetinného místa
\usepackage{multirow}
\usepackage{xtab} %dlouhé  tabulky

\usepackage{lastpage}
\usepackage{fancyhdr} %záhlaví, zápatí
\pagestyle{fancy}
\fancyhead{}
\fancyfoot{}
%\fancyhead[RO,LE]{Aleš FLANDERA}
%\fancyhead[LO,RE]{Praktikum \PRAKTIKUM, Úloha č. \ULOHAN}
\fancyfoot[RO,LE]{\thepage\ z \pageref{LastPage}}
\renewcommand{\headrulewidth}{0.0pt}


\usepackage[fixlanguage]{babelbib}
\selectbiblanguage{czech}
\addto\captionsczech{\renewcommand{\refname}{}} %změna nadpisu literatury (jeho vymazání)

%\usepackage{cite}
%\usepackage{float}
%\usepackage{caption,subcaption}
%\usepackage{longtable}

%%%%%%%%%%%%%%%%%%%%%%%

\begin{document}

\section*{Pravidla}
\hrule
\hfill
\\
\\
Fyziklání online je tříhodinová týmová hra probíhající přes Internet.

Týmy řeší zadané úlohy, řešením každé úlohy je číslo, které odesílají přes 
webový formulář.

Za vyřešené úlohy získávají týmy body, během hry týmy stále vidí aktuální 
výsledkovou listinu, která bude 20 minut před koncem soutěže zmrazena a opět 
aktualizována až po skončení hry.

\section{Účast ve hře}

Týmy ve všech kategoriích řeší stejné úlohy. 

Každá kategorie má samostatnou výsledkovou listinu, bude vytvořena i celková výsledková listina.

Účast ve hře je zdarma.

Je zakázána jakákoliv spolupráce mezi týmy i s osobami mimo tým.

Hra je založena na fair-play. Průběh hry monitorujeme a můžeme tedy odhalit 
některé způsoby podvádění, nicméně předpokládáme, že hrajete proto, abyste si 
zahráli, a tudíž podvádění nemá smysl.

\section{Úlohy}

Zadání úloh budou zveřejněna ve formátu PDF na webu hry po přihlášení 
týmovým heslem.

Výsledkem úlohy je číslo. V zadání úlohy bude, zda požadujeme číslo celé 
(vysloveně) nebo reálné (výchozí). U reálného čísla se bude jeho vyhodnocování 
provádět v rámci intervalu tolerance.

Je-li v zadání uvedena číselná hodnota nějaké konstanty, použijte striktně 
tuto hodnotu.

Číselné výsledky úloh uvádějte v nenásobných jednotkách SI, není-li uvedeno 
jinak. U každého příkladu je uveden minimální počet požadovaných platných cifer, 
ačkoli je možné, že váš výsledek bude správný i při menším počtu zadaných míst, 
jen dodržení vám zaručí správnou opravu výsledku. Příklady:
\begin{itemize}
\item Jakou vzdálenost ujede vlak rychlostí $*50 km.h^{-1}*$ za minutu? Uveďte alespoň 4~platné cifry. — správně je například odpověď 833,3 nebo 833,33

\item Kolik kilometrů ujede vlak rychlostí $*50 km.h^{-1}*$ za minutu? Stačí 1 platná cifra. — například odpověď 0,8

\item Kolik atomů uhlíku (celé číslo) obsahuje molekula cyklopentanoperhydrofenantrenu? — odpověď 17 je správně, odpověď 17,0001 je špatně

\end{itemize}

Úlohy budou rozmanitých typů a rozmanité obtížnosti, která bude rozlišena 
maximem možných získaných bodů, jednodušší úlohy bude obsahovat Hurry-up série, 
ale bude na jejich řešení méně času.

\section{Systém hry a bodování}

Hra bude mít tři části:
\begin{enumerate}
\item První část: 17:00 až 18:30.
\item Druhá část: 18:30 až 20:00.
\item Hurry-up část: 18:00 až 18:30 probíhající souběžně s~první částí v průběhu její poslední půl hodiny.
\end{enumerate}
Konec hry je ve 20:00 (3 hodiny po začátku hry).

V první části bude zadáno každému týmu 5 úloh, po vyřešení úlohy se týmu 
zpřístupní další příklad v sérii.

Za úlohy získávají týmy body. U každé úlohy je určen maximální počet bodů, 
které tým získá, pokud odpoví na úlohy správně napoprvé. Při prvním opakování 
odpovědi získává tým 0,6-násobek maxima, dále analogicky 0,4-násobek a 
0,2-násobek, minimální počet bodů, který získá, je ale 1 bod. Násobky se 
zaokrouhlují nahoru na celý bod.

Při špatné odpovědi bude týmu na 1 minutu znemožněno odpovídat na úlohy dané 
skupiny, tzn. úlohy klasické (první a druhé části), nebo jedné ze tří skupin úloh části Hurry up.

Druhá část na tu první spojitě naváže a její bodování je stejné. Pokračuje 
se řešením stejných příkladů, včetně těch, které již máte k dispozici, popřípadě 
máte rozpracované. Týmy ale navíc získají možnost na nevyřešenou úlohu 
neodpovídat a otevřít si místo ní jinou; za toto bude tým penalizován srážkou jednoho bodu.

V případě rovnosti bodů rozhoduje nižší čas přijetí poslední správné 
odpovědi, v~případě další shody pak rozhodne los.

\subsection*{Hurry-up}

Hodinu po začátku se otevře možnost odpovídat na úlohy tzv. Hurry-up série.

Úlohy této části lze odevzdávat jenom půl hodiny (avšak souběžně s první 
částí).

Úlohy jsou rozděleny na tři tematické oddíly. Na počátku dostane tým z 
každého oddílu jeden příklad. Po správné odpovědi se týmu zpřístupní další úloha 
daného oddílu.

Za každou zkompletovanou trojici (tj. správně zodpovězené úlohy daného 
pořadí z~každého oddílu) bude tým odměněn bonusem ve výši bodů získaných za 
úlohy dané trojice.

Za každou špatnou odpověď na úlohu Hurry-up série se počet bodů získatelných za 
její vyřešení o jeden bod sníží a to až na hranici 0 bodů.

\section{Dovolené vybavení}
Počítač s přístupem na Internet (rychlost připojení by neměla být příliš 
důležitá). Více počítačů může být výhodou (ale není nezbytně nutné). Je dovoleno 
používat Internet jako zdroj informací v libovolné míře.

Tiskárna není nutná, ale může vám pomoci.

Papíry, tužky, pastelky, podložky, literatura, kalkulačka.

\section{Odměny}

V každé kategorii bude odměněno alespoň prvních pět týmů, a to knihami či knižními poukázkami či deskovými nebo karetními hrami.

Odměněné týmy si mohou vybrat ceny z dané množiny, týmy výše v žebříčku mají přednost.

\end{document}
