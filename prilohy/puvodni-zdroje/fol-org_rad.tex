\documentclass[a4paper,11pt,oneside]{article}

\usepackage[IL2]{fontenc}
\usepackage[utf8]{inputenc}
\usepackage[czech]{babel}

\usepackage{fales2}
\usepackage{titulka_a}

\usepackage{geometry} %rozložení stránky

\usepackage{amsfonts,amsmath,amssymb}

\usepackage{graphicx}
\usepackage{color}
\usepackage{epstopdf}

\usepackage{booktabs} %tabulky
\usepackage{array}
\usepackage{dcolumn} %zarovnání tabulek podle desetinného místa
\usepackage{multirow}
\usepackage{xtab} %dlouhé  tabulky

\usepackage{lastpage}
%\usepackage{fancyhdr} %záhlaví, zápatí
%\pagestyle{fancy}
%\fancyhead{}
%\fancyfoot{}
%\fancyhead[RO,LE]{Aleš FLANDERA}
%\fancyhead[LO,RE]{Praktikum \PRAKTIKUM, Úloha č. \ULOHAN}
%\fancyfoot[RO,LE]{\thepage\ z \pageref{LastPage}}


\usepackage[fixlanguage]{babelbib}
\selectbiblanguage{czech}
\addto\captionsczech{\renewcommand{\refname}{}} %změna nadpisu literatury (jeho vymazání)

%\usepackage{cite}
%\usepackage{float}
%\usepackage{caption,subcaption}
%\usepackage{longtable}
\usepackage{enumerate}
\usepackage{hyperref}

%%%%%%%%%%%%%%%%%%%%%%%%%%%%%%%%%%%%%%%%%%%%%%%%%%%%%%%%%%%%%%%%%%%%%%%%%%%%%%%%%%%%%%%%%

\begin{document}
\begin{center}
{\huge{Organizační řád soutěže Fyziklání online}}
%\hfill \linebreak
\end{center}

Univerzita Karlova v~Praze, Matematicko-fyzikální fakulta v~souladu s~\S3 odst. 5 vyhlášky č. 55/2005 Sb., o~podmínkách organizace a~financování soutěží a~přehlídek v~zájmovém vzdělávání, vydává tento organizační řád soutěže Fyziklání online.

\hfill \\

\begin{center}
{\Large{ČÁST PRVNÍ}}\\
{\large{\bf{Základní ustanovení}}}
\end{center}
%
\begin{center}
{\Large{Čl. 1}}\\
\large{\bf{Povaha a~cíl soutěže Fyziklání online}}
\end{center}

\begin{enumerate}[(1)]
\item Fyzikální soutěž Fyziklání online (dále jen \uv{FoL}) je týmová předmětová soutěž z~fyziky, která probíhá v češtině.

\item Cílem je rozvíjení znalostí z~fyziky a~příbuzných oborů a~vyhledávání a~motivování talentovaných studentů.

\item FoL je jednotná soutěž pro území celé České republiky a účastníků mimo Českou republiku, kteří ovládají český jazyk.

\item FoL se člení do kategorií a~má jedno ústřední soutěžní kolo s mezinárodní účastí jednou za školní rok.
\end{enumerate}


\begin{center}
{\Large{Čl. 2}}\\
\large{\bf{Vyhlašovatel}}
\end{center}

\begin{enumerate}[(1)]
\item FoL vyhlašuje Univerzita Karlova v~Praze, Matematicko-fyzikální fakulta  (dále jen \uv{MFF UK}).
\end{enumerate}
%
\hfill \\
\begin{center}
{\Large{ČÁST DRUHÁ}}\\
{\large{\bf{Organizace a~řízení soutěže}}}
\end{center}
%
\begin{center}
{\Large{Čl. 3}}\\
\large{\bf{Organizace}}
\end{center}

\begin{enumerate}[(1)]
\item Proděkan fyzikální sekce MFF UK jmenuje za účelem organizování soutěže FoL Ústřední komisi soutěže Fyziklání online (dále jen \uv{Ústřední komise}) a~to na základě návrhu vedoucího Fyzikálního korespondenčního semináře MFF UK. Ústřední komise je jmenována na dobu určitou a to na dva roky od okamžiku jmenování. Ústřední komisi, popř. její členy, odvolává proděkan fyzikální sekce MFF UK.

\item Ústřední komise je oprávněna přizvat ke spolupráci na organizaci soutěže FoL další subjekty.

\item Ústřední komisí je stanoven termín soutěže pro daný ročník, a~to vždy v~průběhu příslušného školního roku.

\item Na začátku školního roku je Ústřední komisí vyhlášen harmonogram soutěže a~pravidla pro daný ročník.

\item FoL je rozděleno do tří kategorií. Studenti mladší prvního ročníku  středních škol a odpovídajících ročníků víceletých gymnázií si přiřadí koeficient hráče 0, studenti prvního ročníku koeficient hráče 1, druhého 2, atd. koeficient týmu se spočte jako průměrná hodnota koeficientů hráčů (koeficienty hráče od jednotlivých členů se sečtou a~vydělí počtem členů týmu). Tým se zařadí do nejnižší kategorie, která mu vyhovuje: 
\begin{enumerate}[({kategorie} A)]
\item koeficient týmu $\leq$ 4,
\item koeficient týmu $\leq$ 3 a~max. dva členové týmu mají koeficient hráče 4,
\item koeficient týmu $\leq$ 2 a~žádný člen týmu nemá koeficient hráče 4 a~max. dva členové týmu mají koeficient hráče 3,
\end{enumerate}
kategorie A~je nejvyšší, kategorie C je nejnižší.

\item Sporné situace v~průběhu konání soutěže FoL rozhoduje Ústřední komise.

\item Ústřední komise má právo na zásah do konání soutěže včetně vyřazení úlohy, pokud je to v~zájmu její regulérnosti.
\end{enumerate}

\begin{center}
{\Large{Čl. 4}}\\
\large{\bf{Účast v~soutěži}}
\end{center}

\begin{enumerate}[(1)]
\item Účast v~soutěži FoL je dobrovolná.

\item Přihláškou do soutěže účastník, resp. jeho zákonný zástupce, souhlasí s~tímto organizačním řádem a~zavazuje se jím řídit. Zavazuje se též řídit aktuálními pravidly soutěže.

\item Dále svou přihláškou do soutěže souhlasí s~uvedením jména, příjmení, počtu bodů jeho soutěžního týmu a~u~žáků také školy na veřejných výsledkových listinách.

\item Svou přihláškou do soutěže každý účastník souhlasí se zpracováváním svých údajů, a~to konkrétně jména, adresy, emailové adresy, roku narození, roku maturity a~školy za účelem organizace soutěže a~Fyzikálního korespondenčního semináře MFF UK. Tento souhlas může účastník kdykoliv odvolat.  S osobními údaji bude nakládáno v souladu se z.č. 101/2000 Sb., ve znění pozdějších předpisů.

\item Vyplněním přihlášky registrátor či účastníci stvrzují správnost poskytnutých údajů. V případě, že bude přihláška obsahovat nesprávné či nepravdivé údaje, je Ústřední komise oprávněna  tým, který takovou přihlášku podal, vyřadit z výsledkové listiny a to i dodatečně po ukončení soutěže.

\item Každý soutěžní tým má minimálně jednoho, maximálně pak pět členů týmu.

\item Veškeré náklady spojené s účastí v soutěži FoL nese každý účastník v plném rozsahu sám.

\item Přihlašování do soutěže se řídí aktuálními pravidly soutěže.

\item Účast žáků v~soutěži je možné považovat za činnost, která přímo souvisí s~vyučováním.

\end{enumerate}

\begin{center}
{\Large{Čl. 5}}\\
\large{\bf{Konání soutěže}}
\end{center}

\begin{enumerate}[(1)]
\item Podrobná pravidla soutěže FoL připravuje a~schvaluje Ústřední komise.

\item Úkolem soutěžících je získat co nejvíce bodů během časového limitu stanoveného pravidly soutěže. Tým, který získá nejvíce bodů ve své kategorii se stává vítězným týmem.

\item Pořadí týmů je pravidly určeno jednoznačně.

\item Odevzdávání úloh se řídí pravidly soutěže.

\item Soutěžní úlohy jsou utajeny a~zveřejněny na soutěžním webu se začátkem soutěže podle pravidel soutěže.

\item Správnost výsledků vyhodnocuje počítačový systém.

\item Úspěšným týmem soutěže se stává ten tým, který dosáhl alespoň 60\% z~maximálního počtu bodů a~zároveň se umístil do desátého místa ve své kategorii včetně. Maximální počet bodů je stanoven podle počtu bodů, kterého dosáhne vítězný
tým.

\item Výsledky soutěže budou ve stanoveném termínu zveřejněny na adrese \\ \url{http://online.fyziklani.cz} a~předány ministerstvu podle \S{}8 vyhlášky č. 55/2005 Sb. 

\item Na věcné a~jiné odměny nevzniká účastníkům právní nárok. 

\end{enumerate}
\hfill \\

\pagebreak
\begin{center}
{\Large{ČÁST TŘETÍ}}\\
%{\large{\bf{}}}
\end{center}
%
\begin{center}
{\Large{Čl. 6}}\\
\large{\bf{Bezpečnost a~ochrana zdraví}}
\end{center}

\begin{enumerate}[(1)]
\item Ústřední komise ani MFF UK nijak nezodpovídá za bezpečnost a~ochranu zdraví účastníků během soutěže.
\end{enumerate}

\begin{center}
{\Large{Čl. 7}}\\
\large{\bf{Závěrečná ustanovení}}
\end{center}

\begin{enumerate}[(1)]
\item Tento organizační řád byl schválen děkanem MFF UK 25.\,4.\,2013.

\item Tento organizační řád nabývá účinnosti dne 25.\,4.\,2013.
\end{enumerate}

\begin{center}
{\Large{Čl. 8}}\\
\large{\bf{Přechodná ustanovení}}
\end{center}

\begin{enumerate}[(1)]
\item V případě zařazení soutěže FoL do dotačního programu Podpora soutěží a přehlídek v zájmovém vzdělávání vyhlašovaném Ministerstvem školství, mládeže a tělovýchovy se dnem zařazení do tohoto dotačního programu mění znění bodu (1) Čl. 2 na: \uv{FoL vyhlašuje Ministerstvo školství, mládeže a tělovýchovy společně s Univerzitou Karlovou v Praze, Matematicko-fyzikální fakultou (dále jen \uv{MFF UK})}.
\end{enumerate}

\end{document}